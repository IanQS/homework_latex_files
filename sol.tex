%%%%%%%%%%%%%%%%%%%%%%%%%%%%%%%%%%%%%%%%%
% Cleese Assignment (For Students)
% LaTeX Template
% Version 2.0 (27/5/2018)
%
% This template originates from:
% http://www.LaTeXTemplates.com
%
% Author:
% Vel (vel@LaTeXTemplates.com)
%
% License:
% CC BY-NC-SA 3.0 (http://creativecommons.org/licenses/by-nc-sa/3.0/)
% 
%%%%%%%%%%%%%%%%%%%%%%%%%%%%%%%%%%%%%%%%%

%----------------------------------------------------------------------------------------
%	PACKAGES AND OTHER DOCUMENT CONFIGURATIONS
%----------------------------------------------------------------------------------------

\documentclass[11pt]{article}

\input{sol_utils_and_boilerplate.tex} % Include the file specifying the document structure and custom commands

%----------------------------------------------------------------------------------------
%	ASSIGNMENT INFORMATION
%----------------------------------------------------------------------------------------

% Required
\newcommand{\assignmentClassCode}{CLASS-CODE} % Course/class
\newcommand{\classTitle}{CLASS-NAME}
\newcommand{\assignmentTitle}{X} % Assignment title or name
\newcommand{\assignmentQuestionName}{Question} % The word to be used as a prefix to question numbers; example alternatives: Problem, Exercise

% You
\newcommand{\studentName}{Ian Quah} % Student name
\newcommand{\studentNumber}{2330981} % Student name

% Optional (comment lines to remove)
\newcommand{\assignmentDueDate}{} % Due date
\newcommand{\studentID}{itq} % Student name

%----------------------------------------------------------------------------------------

\begin{document}

%----------------------------------------------------------------------------------------
%	TITLE PAGE
%----------------------------------------------------------------------------------------

\maketitle % Print the title page

\thispagestyle{empty} % Suppress headers and footers on the title page

%----------------------------------------------------------------------------------------
%	QUESTION 1
%----------------------------------------------------------------------------------------

\begin{question}

\questiontext{Main Question}


\begin{subquestion}{Sub-question 1} % Subquestion within question
\answer{
	
}
\end{subquestion}
\end{question}

%----------------------------------------------------------------------------------------
%	QUESTION 2
%----------------------------------------------------------------------------------------
\begin{question}

\questiontext{Some Python Code}

\begin{minted}{python}
import numpy as np
import matplotlib.pyplot as plt
from typing import Callable

def adder(x):
	return x + 1

\end{minted}
\end{question}

%----------------------------------------------------------------------------------------
%	QUESTION 4
%----------------------------------------------------------------------------------------
\begin{question}

\begin{table}[h]
	\centering % Centre the table
	\begin{tabular}{l l l}
		\toprule
		\textit{Per 50g} & Pork & Soy \\
		\midrule
		Energy & 760kJ & 538kJ\\
		Protein & 7.0g & 9.3g\\
		Carbohydrate & 0.0g & 4.9g\\
		Fat & 16.8g & 9.1g\\
		Sodium & 0.4g & 0.4g\\
		Fibre & 0.0g & 1.4g\\
		\bottomrule
	\end{tabular}
\end{table}

\answer{Lorem ipsum dolor sit amet, consectetur adipiscing elit. Praesent porttitor arcu luctus, imperdiet urna iaculis, mattis eros. Pellentesque iaculis odio vel nisl ullamcorper, nec faucibus ipsum molestie. Sed dictum nisl non aliquet porttitor. Etiam vulputate arcu dignissim, finibus sem et, viverra nisl. Aenean luctus congue massa, ut laoreet metus ornare in. Nunc fermentum nisi imperdiet lectus tincidunt vestibulum at ac elit. Nulla mattis nisl eu malesuada suscipit.}

\end{question}

\end{document}
